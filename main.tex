\documentclass[11.5pt]{article}
\usepackage[ngerman]{babel}
\usepackage{hyphenat}

\usepackage[backend=biber, style=nature, sorting=ynt]{biblatex}
\usepackage{csquotes}
\usepackage{textgreek}
\addbibresource{main.bib}

\title{Diagnose und Klassifikation von \textit{Diabetes mellitus} mithilfe maschinellen Lernens}
\author{Nico Piel \\ 9034834 \\ nico.piel@hotmail.de}
\date{7. Januar 2020}
\usepackage[onehalfspacing]{setspace}

\begin{document}

\maketitle

\newpage

\tableofcontents

\newpage

\begin{abstract}

Ziel dieser Arbeit war es, mithilfe maschineller Lernmethoden und klinischen Daten zu überprüfen, ob man \textit{Diabetes mellitus} ohne Gensequenzierung diagnostizieren und klassifizieren kann. 
Mittels einer Support Vector Machine und einem neuronalen Netz soll eine Klassifikation erfolgen.
% TODO: Ergebnisse präsentieren
Interessant ist diese Arbeit vor allem für Studenten der Bioinformatik.

\end{abstract}

\newpage

\section{Diabetes mellitus}


\subsection{Definition}


Diabetes ist eine Gruppe von Stoffwechselerkrankungen, die von der sogenannten Hyperglykämie, einer krankhaft erhöhten Blutzuckerkonzentration, charakterisiert werden. Grund dafür sind Defekte bei der Wirkung oder Sekretion des in der Bauchspeicheldrüse produzierten Proteohormons Insulin \cite[p.~62]{ada}. Durch diese chronische Hyperglykämie können verschiedene Langzeitkomplikationen entstehen, unter anderem Dysfunktionen und Organversagen der Augen, Leber, Nerven, des Herzen und der Blutgefäße \cite[p.~540]{who}.
Es gibt eine Reihe an Krankheitserregern; besonders nennenswert sind der absolute Insulinmangel durch autoimmune Zerstörung der \textbeta-Zellen in der Bauchspeicheldrüse und verschiedene Abnormalitäten des Stoffwechsels, die zu Insulinresistenz führen. Grund für diese Stoffwechselprobleme sind mangelhafte oder fehlende Wirkung - beziehungsweise unzureichende Aufnahme - des Insulin, oft bedingt durch unzulängliche Sekretion des Hormons. Häufig koexistieren Wirkungs- und Sekretionsprobleme im selben Patienten, sodass unklar ist, welches der beiden, wenn nicht beide, die Hyperglykämie verursachen.

\subsection{Symptomatik, Direkt- und Langzeitfolgen}


Symptome der Hyperglykämie sind unter anderem Polyurie, Polydipsie, Gewichtsverlust, manchmal Polyphagie und Sehbeeinträchtigungen\footnotemark \cite[p.~62]{ada}. Bei besonders frühen oder schweren Erkrankungen kann es auch zu Wachstumsstörungen oder Schwächungen des Immunsystems kommen. Akute, lebensgefährliche Direktfolgen eines unkontrollierten Diabetes sind Ketoazidose durch Langzeithyperglykämie und hyperosmolares Koma\footnotemark[\value{footnote}] \cite[p.~540]{who}.


Selbst durch einen gut behandelten Diabetes können Langzeitkomplikationen entstehen, darunter Retinopathie und eventuelle Blindheit; Nephropathie und daraus folgendes Nierenversagen; periphere Neuropathie mit Risiko auf Fußgeschwüre, Amputationen oder Charcot-Füße; und gastrointestinale, urogenitale und kardiovaskulare Symptome und sexuelle Dysfunktion\footnotemark[\value{footnote}]. Patienten  mit Diabetes haben eine erhöhte Inzidenz für die koronare Herzkrankheit, periphere arterielle  Verschlusskrankheit und zerebrovaskuläre Krankheiten. Auch Hypertonie und Störungen des Lipoproteinstoffwechsels kommen bei Patienten mit Diabetes häufig vor\footnotemark[\value{footnote}]. \cite[p.~62]{ada}

\footnotetext{Definition der Fachbegriffe im Anhang}
\newpage


\subsection{Klassifikation}


Der Großteil von Diabetespatienten lässt sich in zwei weite ätiologische Kategorien unterteilen. Nur ein verschwindend geringer Teil der Erkrankten lässt sich nicht in eine der beiden unterteilen, weil die Ursache unbekannter Herkunft oder sehr speziell ist.

\subsubsection{Diabetes mellitus Typ 1}


Bei \textit{Typ 1 Diabetes} (vormals \glqq insulinabhängiger Diabetes\grqq{}) liegt die Quelle der Krankheit in einem absoluten Insulinmangel. Der Körper zerstört die \textbeta-Zellen in den Langerhans-Inseln der Bauchspeicheldrüse durch eine zelluläre Immunantwort der cytotoxischen T-Zellen \cite[p.~62]{ada}. Diese Form der Krankheit betrifft etwa 5-10\% der Patienten. 

\subsubsection{Diabetes mellitus Typ 2}


\textit{Typ 2 Diabetes} (vormals auch \glqq nicht-insulinabhängiger Diabetes genannt\grqq{}) betrifft etwa 90-95\% der Erkrankten. Die Betroffenen entwickeln eine Insulinresistenz und häufig auch relativen (statt absoluten) Insulinmangel. Diese Patienten benötigen häufig zu Beginn und auch während ihres Lebens \textbf{keine} Insulintherapie. \cite[p.~63]{ada}

\newpage

\subsection{Diagnose}


Die für diese Arbeit wichtigsten Voraussetzungen sind die Kriterien zur Diagnose der Krankheit. Entgegen der Intuition, reicht ein stark erhöhter Blutzucker alleine nicht, denn auch andere Faktoren nehmen Einfluss auf diesen Wert \cite[p.~540]{who}. So führen zum Beispiel Infektionen, Stress, Traumata und Kreislaufprobleme ebenfalls zu einem temporär erhöhten Blutzuckerspiegel \cite[p.~540]{who}


\newpage

\appendix
\section{Anhang}


\newpage

\printbibliography

\end{document}
